\documentclass[oneside,12pt]{book}
\usepackage[utf8]{inputenc}
\usepackage{standalone}
\usepackage[margin=2.5cm]{geometry}
\usepackage{graphicx}
\usepackage{hyperref}
\usepackage{lmodern}
\usepackage{float}
\usepackage[spanish]{babel}
\usepackage{csquotes}
\usepackage[figurename=Ilustración]{caption}
\usepackage[numbers]{natbib}
\usepackage{ragged2e}
\usepackage{array}
\usepackage{multirow}
\usepackage{parskip}
\usepackage{listings}
\usepackage{epigraph}
\usepackage{lipsum} % Texto de relleno en plantilla

\usepackage{comment} %Uso de comentarios multilínea

%\usepackage{fancyhdr}
\usepackage{geometry}
\usepackage{color}\usepackage{graphicx,xcolor}

\usepackage{titlesec}
\setcounter{secnumdepth}{3}
\titleformat{\chapter}[display]
{\normalfont\huge\bfseries}{\chaptertitlename\ \thechapter}{20pt}{\Huge}

\urlstyle{same}
\title{Galería de NFT's en la Web3}
\author{Cristóbal José Jiménez Gómez}
\date{Curso 2022 - 2023}
\renewcommand{\contentsname}{Contenido}
\renewcommand{\figurename}{Ilustración}
\usepackage{pifont}
\renewcommand\labelitemi{\ding{52}}

%Agregados por mi%
%para la bibliografía%
\usepackage{natbib}
\bibliographystyle{unsrtnat}

%Para las tablas
\usepackage{tabularx}

%Para las imágenes
\usepackage{graphicx}

%Colores 
\usepackage{xcolor}

\begin{document}
\begin{titlepage}
    \begin{center}
    \vspace*{0.5in}
        \begin{figure}[htb]
            \begin{center}
            \includegraphics[width=15cm]{Ilustraciones/LogoEII.jpg}
            \end{center}
        \end{figure}

        \vspace*{0.15in}
        \vspace*{0.2in}

        \noindent\hfil\rule{17cm}{0.2mm}\hfil\\
        \vspace*{0.1in}
        \begin{Huge}
            \textbf{Galería de NFT's en Web3} \\
        \end{Huge}

        \vspace*{0.3in}

        \begin{large}
            TITULACIÓN: Grado en Ingeniería Informática \\
            \vspace*{0.1in}
            AUTOR: Cristóbal José Jiménez Gómez \\
        \end{large}
        \vspace*{0.3in}
        
        \noindent\hfil\rule{17cm}{0.2mm}\hfil\\

        \vspace*{0.1in}

        \begin{large}
            TUTORIZADO POR: \\
            María Dolores Afonso Suárez \\
        \end{large}
        
    \vspace*{0.3in}

    Fecha: 05/2023
    \end{center}
\end{titlepage}


%\input{TFT04C.tex}

\newpage
\pagenumbering{arabic}
{\Large{\textbf{Agradecimientos}}}
\input{Agradecimientos.tex}

\clearpage
{\setlength{\parskip}{8mm}
{\huge{\textbf{Resumen}}}

    {\large{ 
        Desarrollo de una Aplicación Web en el marco de la Web3(blockchain), que gestiona una galería de 
        Non Fungible Tokens (NFT’s). Ofrece la gestión de contenidos a distintos tipos de usuarios cuyos niveles de
        interacción varían desde los usuarios sin registrar, a administradores. En esa escala de accesos los
        casos de uso establecerán el nivel de interacción de cada perfil. Las funcionalidades implementadas
        permitirán la gestión a distintos niveles de estos NFT’s, desde la visualización hasta la gestión de cada
        cartera. Este último concepto es el definido para almacenarlos y, en caso de ser necesario comerciar
        (compraventa o intercambio) con ellos.
    }}

    \textbf{Palabras claves:} Web3, Web3.0, Angular, Firebase, Vercel, Express
    

%\newpage

{\huge{\textbf{Abstract}}}

    {\large{
        Development of a Web Application within the framework of Web3 (blockchain), which manages a gallery of
        Non Fungible Tokens (NFT's). It offers content management to different types of users whose levels of
        Interaction range from unregistered users to administrators. At this access scale, the
        use cases will establish the level of interaction of each profile. The functionalities implemented
        will allow the management at different levels of these NFTs, from the visualization to the management of each
        briefcase. This last concept is the one defined to store them and, in case it is necessary to trade
        (purchase or exchange) with them.
    }}

    \textbf{Key Words:} Web3, Web3.0, Angular, Firebase, Vercel, Express
} 
\tableofcontents
\newpage
\listoffigures
\newpage
\listoftables

\clearpage
\pagenumbering{arabic}

%----Introducción----%
%--Motivación/Objetivos/Estructura de memoria--%
\chapter{Introducción}
\section{Motivación}
La razón principal que se encontró para la elaboración de este trabajo ha sido el auge de la Blockchain\cite{whatIsWeb3} durante estos últimos años.

\section{Objetivos}
hacer cosas

\section{Estructura de la memoria}
Cosas


\newpage
%----Competencias----%
\chapter{Competencias específicas}
Las competencias aplicadas a este proyecto se pueden encontrar en la Tabla 
\ref{tab:competencias} y a continuación se listan: CI8, CI13, CI16, CI17, T12.\\
\-\hspace{1cm} La competencia CI8 se justifica debido a la necesidad de analizar,
diseñar y construir los distintos componentes de la página web, así como las 
estructuras de los distintos documentos de la base de datos.\\
\-\hspace{1cm} La competencia CI13 se justifica por el uso de hacer llamadas a API's
y la creación de un backend. Así como a la hora de desplegar la aplicación a internet. \\
\-\hspace{1cm} La competencia CI16 se justifica debido a la metodología Scrum aplicada
durante el transcurso del proyecto. \\
\-\hspace{1cm} La competencia CI17 se justifica gracias al uso de los programas de 
caldiad utilizados para garantizar la accesibilidad y usabilidad de la página web.\\
\-\hspace{1cm} La competencia T12 se justifica debido a que en eso ha consistido este proyecto: \\
\-\hspace{1.5cm} \textbf{Diseño} de los distintos componentes y páginas de la aplicación.\\
\-\hspace{1.5cm} \textbf{Despliegue} de la aplicación en sus diferentes ramas para poder ser \textbf{evaluadas} 
    por distintos usuarios.\\
\-\hspace{1.5cm} \textbf{Selección} de las herramientas como pueden ser el \textit{framework} sobre el que se ha 
trabajado (Angular), o la página sobre la que se ha hecho el \textit{despliegue}, siempre teniendo en \textit{cuenta el coste 
y la calidad} del mismo.\\

\begin{table}[H]
    \caption{\textbf{Competencias} cubiertas durante el desarrollo del proyecto.}
    \hfill \break
    \label{tab:competencias}
    \begin{tabularx}{\linewidth}{| @{} >{\bfseries}l | >{\RaggedRight}X @{} | }
        \hline
        \centering \textbf{Código} & \textbf{Descripción} \\
        \hline
        CI8  & Capacidad para analizar, diseñar, construir y mantener 
        aplicaciones de forma robusta, segura y eficiente, eligiendo el 
        paradigma y los lenguajes de programación más adecuados. \\ 
        \hline
        CI13 & Conocimiento y aplicación de las herramientas necesarias para el 
        almacenamiento, procesamiento y acceso a los sistemas de 
        información, incluidos los basados en web. \\ 
        \hline
        CI16 & Conocimiento y aplicación de los principios, metodologías y ciclos 
        de vida de la ingeniería del Software \\
        \hline
        CI17 & Capacidad para diseñar y evaluar interfaces persona computador que 
        garanticen la accesibilidad y usabilidad a los sistemas, servicios y 
        aplicaciones informáticas. \\ 
        \hline
        T12  & Capacidad para seleccionar, diseñar, desplegar, integrar, evaluar, 
        construir, gestionar, explotar y mantener las tecnologías 
        de hardware, software y redes, dentro de los parámetros de coste y 
        calidad adecuados.\\ 
        \hline
    \end{tabularx}
    \hfill \break
    \hfill \break
    \centering\textbf{Fuente:} Universidad de Las Palmas de Gran Canaria (2023)
\end{table}
\newpage
%----Estado actual y Objetivos iniciales----%
%--Objetivos iniciales/Estado actual--%
\chapter{Objetivos iniciales y estado actual}
\section{Historia de la Web hasta la Web 3.0}
La \textbf{World Wide Web} fue creada en 1989 por el científico británico 
\textit{Tim Berners-Lee} mientras trabajaba en el \textit{Centro Europeo de Investigación 
Nuclear(CERN)}. La idea era crear una red de información que pudiera ser compartida 
entre científicos y académicos de todo el mundo. Para evitar un apagado accidental se escribió 
una nota en tinta roja que ponía: "\textcolor{red}{This machine is a server. DO NOT 
POWER IT DOWN!!}"\cite{cernWeb1} (Esta máquina es un servidor. ¡¡NO LO APAGUEN!!) \\
\hfill \break
In 1991, \textit{Berners-Lee} creó una serie de tecnologías que permitían la 
conexión de documentos en un sistema hipertextual, utilizando el 
\textit{protocolo \textbf{HTTP}} y la \textit{codificación \textbf{HTML}}. 
Esto permitió a los usuarios navegar por la red y acceder a documentos 
enlazados desde cualquier parte del mundo. \\
\hfill \break
Con el tiempo, la Web se expandió y evolucionó, surgieron nuevas tecnologías como los 
motores de búsqueda, las redes sociales y las aplicaciones móviles, lo que llevó
a la denominada \textbf{Web 2.0}.\\
\hfill \break
La Web 2 se caracterizó por una mayor interactividad, el desarrollo de 
aplicaciones colaborativas y la creación de plataformas para la participación 
del usuario, como blogs, wikis y redes sociales. La Web2 también permitió la 
creación de empresas en línea y el desarrollo de nuevos modelos de negocio basados 
en la publicidad y los servicios en línea.\cite{web2Explained}\\
\hfill \break
La \textbf{Web 3}, también conocida como \textit{Web descentralizada}, 
es la siguiente evolución de la Web. La Web 3.0 se centra en la descentralización 
de la web, lo que significa que los usuarios tienen mayor control sobre sus 
datos y pueden interactuar directamente entre sí sin la necesidad de 
intermediarios centralizados.\\
\hfill \break
La tecnología clave detrás de la Web 3 es la cadena de bloques (blockchain) y 
otras tecnologías de registro distribuido (DLT), que permiten la creación de 
aplicaciones descentralizadas (dApps) y contratos inteligentes (smart contracts). 
Esto permite la creación de aplicaciones que no estén sujetas a la censura, la 
interferencia o la dependencia de un solo proveedor, lo que a su vez ofrece 
mayor privacidad y seguridad para los usuarios.\\
\hfill \break
En resumen, la Web ha evolucionado desde su creación como WWW hasta la 
actualidad de la Web3. La Web 3 representa una evolución hacia un internet 
más descentralizado y democrático, donde los usuarios tienen un mayor control 
sobre sus datos y su experiencia en línea, y donde la confianza y la seguridad 
se pueden garantizar a través de la tecnología blockchain y otros mecanismos 
de confianza descentralizados.\cite{IEEEweb3Explained}\\

\section{Concepto de Cryptomoneda}
\input{./3.-Marco-Teorico/3.2-criptomonedas.tex}

\section{Concepto de NFT}
\input{./3.-Marco-Teorico/3.3-nft.tex}

\section{Concepto de CryptoCartera}
\input{./3.-Marco-Teorico/3.4-criptowallet.tex}
\newpage
%----Desarrollo como concepto----%
\begin{comment}
        CAPÍTULO OBLIGATORIO
        Metodología aplicada y desarrollo del trabajo en sus distintas fases, decisiones de diseño, herramientas empleadas.
        Si el trabajo incluye software desarrollado, deberán seleccionarse las secciones más relevantes del mismo y comentarlas en la memoria.
        Ajuste a la planificación inicialmente prevista.
        Modificación en los objetivos planteados.
\end{comment}
\chapter{Marco teórico}
\newpage
\chapter{Recursos y tecnologías}
\newpage
\chapter{Metodologías}
\newpage
\chapter{Análisis y Diseño}
\newpage
\chapter{Desarrollo}
\newpage
%----Fin del Desarrollo----%

%----Evaluación y resultados----%
\chapter{Evaluación y resultados}

%----Conclusiones y trabajo futuro----%

\chapter{Conclusiones y trabajo futuro}
%\input{Conclusiones.tex}
CAPÍTULO OBLIGATORIO

Resultados,  grado  de  consecución  de  los  objetivos,  posibles extensiones
\newpage

\bibliography{bib.bib}

\end{document}