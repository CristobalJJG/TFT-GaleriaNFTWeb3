\documentclass[oneside,12pt]{book}
\usepackage[utf8]{inputenc}
\usepackage{standalone}
\usepackage[margin=2.5cm]{geometry}
\usepackage{graphicx}
\usepackage{hyperref}
\usepackage{lmodern}
\usepackage{float}
\usepackage[spanish]{babel}
\usepackage{csquotes}
\usepackage[figurename=Ilustración]{caption}
\usepackage[numbers]{natbib}
\usepackage{ragged2e}
\usepackage{array}
\usepackage{multirow}
\usepackage{parskip}
\usepackage{listings}
\usepackage{epigraph}
\usepackage{lipsum} % Texto de relleno en plantilla

\usepackage{comment} %Uso de comentarios multilínea

%\usepackage{fancyhdr}
\usepackage{geometry}
\usepackage{color}\usepackage{graphicx,xcolor}

\renewcommand{\lstlistingname}{Algoritmo}
\renewcommand{\lstlistlistingname}{Índice de~\lstlistingname s}
\lstset{
        basicstyle=\fontsize{7}{11}\ttfamily,
        language=Java,
        captionpos=b,
}

\usepackage{titlesec}
\setcounter{secnumdepth}{3}
\titleformat{\chapter}[display]
{\normalfont\huge\bfseries}{\chaptertitlename\ \thechapter}{20pt}{\Huge}

\urlstyle{same}
\title{Galería de NFT's en la Web3}
\author{Cristóbal José Jiménez Gómez}
\date{Curso 2022 - 2023}
\renewcommand{\contentsname}{Contenido}
\renewcommand{\figurename}{Ilustración}
\usepackage{pifont}
\renewcommand\labelitemi{\ding{52}}

\begin{document}
\begin{titlepage}
    \begin{center}
    \vspace*{0.5in}
        \begin{figure}[htb]
            \begin{center}
            \includegraphics[width=15cm]{Ilustraciones/LogoEII.jpg}
            \end{center}
        \end{figure}

        \vspace*{0.15in}
        \vspace*{0.2in}

        \noindent\hfil\rule{17cm}{0.2mm}\hfil\\
        \vspace*{0.1in}
        \begin{Huge}
            \textbf{Galería de NFT's en Web3} \\
        \end{Huge}

        \vspace*{0.3in}

        \begin{large}
            TITULACIÓN: Grado en Ingeniería Informática \\
            \vspace*{0.1in}
            AUTOR: Cristóbal José Jiménez Gómez \\
        \end{large}
        \vspace*{0.3in}
        
        \noindent\hfil\rule{17cm}{0.2mm}\hfil\\

        \vspace*{0.1in}

        \begin{large}
            TUTORIZADO POR: \\
            María Dolores Afonso Suárez \\
        \end{large}
        
    \vspace*{0.3in}

    Fecha: 05/2023
    \end{center}
\end{titlepage}


%\input{TFT04C.tex}

\newpage
\pagenumbering{arabic}
{\Large{\textbf{Agradecimientos}}}
\input{Agradecimientos.tex}

\clearpage
{\setlength{\parskip}{8mm}
{\huge{\textbf{Resumen}}}

    {\large{ 
        Desarrollo de una Aplicación Web en el marco de la Web3(blockchain), que gestiona una galería de 
        Non Fungible Tokens (NFT’s). Ofrece la gestión de contenidos a distintos tipos de usuarios cuyos niveles de
        interacción varían desde los usuarios sin registrar, a administradores. En esa escala de accesos los
        casos de uso establecerán el nivel de interacción de cada perfil. Las funcionalidades implementadas
        permitirán la gestión a distintos niveles de estos NFT’s, desde la visualización hasta la gestión de cada
        cartera. Este último concepto es el definido para almacenarlos y, en caso de ser necesario comerciar
        (compraventa o intercambio) con ellos.
    }}

    \textbf{Palabras claves:} Web3, Web3.0, Angular, Firebase, Vercel, Express
    

%\newpage

{\huge{\textbf{Abstract}}}

    {\large{
        Development of a Web Application within the framework of Web3 (blockchain), which manages a gallery of
        Non Fungible Tokens (NFT's). It offers content management to different types of users whose levels of
        Interaction range from unregistered users to administrators. At this access scale, the
        use cases will establish the level of interaction of each profile. The functionalities implemented
        will allow the management at different levels of these NFTs, from the visualization to the management of each
        briefcase. This last concept is the one defined to store them and, in case it is necessary to trade
        (purchase or exchange) with them.
    }}

    \textbf{Key Words:} Web3, Web3.0, Angular, Firebase, Vercel, Express
} 
\tableofcontents
\listoffigures
\listoftables
\lstlistoflistings

\clearpage
\pagenumbering{arabic}

\chapter{Introducción}
\section{Motivación}
La razón principal que se encontró para la elaboración de este trabajo ha sido el auge de la Blockchain\cite{whatIsWeb3} durante estos últimos años.

\section{Objetivos}
hacer cosas

\section{Estructura de la memoria}
Cosas



%
\chapter{Estado actual y objetivos iniciales}
Estado del arte to rechulón
CAPÍTULO OBLIGATORIO

Descripción del estado actual en la temática concreta del trabajo y desglose de los objetivos inicialmente previstos en el TFT01.

\chapter{Competencias específicas y aportaciones del trabajo}
%\input{Competencias.tex}
CAPÍTULO OBLIGATORIO

Indicar, sólo para las competencias específicas relacionadas de forma más directa con el trabajo desarrollado, cómo se han cubierto con este TFT.

Justificar qué es lo que este TFT aporta a nuestro entorno socio-económico, técnico o científico.

\chapter{Desarrollo}
%\input{Desarrollo.tex}
CAPÍTULO OBLIGATORIO

Metodología aplicada y desarrollo del trabajo en sus distintas fases, decisiones de diseño, herramientas empleadas.

Si el trabajo incluye software desarrollado, deberán seleccionarse las secciones más relevantes del mismo y comentarlas en la memoria.

Ajuste a la planificación inicialmente prevista.

Modificación en los objetivos planteados.

\chapter{Conclusiones y trabajo futuro}
%\input{Conclusiones.tex}
CAPÍTULO OBLIGATORIO

Resultados,  grado  de  consecución  de  los  objetivos,  posibles extensiones

\chapter{OTROS}

El documento debe terminar con las referencias bibliográficas (SECCIÓN OBLIGATORIA). Después pueden incluirse anexos 

CAPÍTULOS OPCIONALES

Dependiendo del tipo de trabajo, se podrían incluir capítulos adicionales como los siguientes:

\begin{description}
\item{REQUISITOS}
\item{DISEÑO}
\item{MANUAL  DE  USUARIO  Y  SOFTWARE}  deberán  incluirse  obligatoriamente  en  la  memoria  los extractos más relevantes   del código desarrollado. Siempre que sea posible, deberá   proporcionarse acceso a un repositorio software.
\item{NORMATIVA Y LEGISLACIÓN} incluir la legislación vigente que afecte al TFT (ley de protección de datos, leyes sobre seguridad, ...)
\item{ASPECTOS ECONÓMICOS Y TEMPORALES}
\end{description}



\bibliography{ref}
\bibliographystyle{apalike}

\end{document}