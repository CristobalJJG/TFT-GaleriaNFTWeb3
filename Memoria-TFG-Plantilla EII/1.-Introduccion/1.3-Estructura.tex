La memoria está dividida en 10 capítulos diferentes. En ellos se recoge toda
la información, desde el inicio del proyecto hasta las conclusiones. Así, 
la información recogida en cada capítulo se describe tal que:

\-\hspace{1cm} El primer capítulo consta de la introducción al proyecto. En este se 
contextualiza el trabajo, se explica la motivación detrás de la elaboración 
del proyecto, se dan a conocer los objetivos a cumplir y se explica la 
estructura del documento.

\-\hspace{1cm} El segundo capítulo lista y justifica las competencias de la titulación
abarcadas durante el desarrollo del trbajo mediante las teras y actividades realizadas.

\-\hspace{1cm} El tercer capítulo desarrolla el marco teórico, donde se explican y 
analizan algunos términos relevantes a este trabajo. Es el punto de partida, 
donde se realiza una investifación que contextualiza y justifica las cacciones
que se toman en el trabajo.

\-\hspace{1cm} El cuarto capítulo abarca el estado del arte, que explica la 
actualidad del desarrollo Web junto con el desarrollo dentro de la Web3.0 y 
la popularidad entre los distintos frameworks\footnote{Un framework es una 
herramienta de programación que te permite desarrollar software proporcionando 
una estructura con componentes integrados que sirven de base para construir 
proyectos nuevos\cite{bootcampFramework}}

\-\hspace{1cm} El quinto capítulo contiene los recursos y las tecnologías 
utilizadas para el desarrollo del trabajo. En este se definen y justifican.

\-\hspace{1cm} El sexto capítulo se explica la metodología llevada a cabo 
para la realización del trabajo, se dan a conocer als diferentes fases y 
tareas que han existido durante su desarrollo.

\-\hspace{1cm} El séptimo capítulo se incluye un análisis de los requisistos y 
el diseño de la página que se ha elaborado para el desarrollo.

\-\hspace{1cm} El octavo capítulo se explica todo lo relacionado con el 
desarrollo de la página web.

\-\hspace{1cm} El noveno capítulo se encuentra la evaluación de la calidad de la página, 
obtenidos a través de diversos programas o extensiones.

\-\hspace{1cm} El décimo capítulo se incluye las conclusiones que se pueden obtener 
de este proyecto, tanto personales como objetivas, dando además una propuesta de mejora 
para un futuro desarrollo.

\-\hspace{1cm} Finalmente se incluyen los anexos, los cuales también contienen las 
referencias bibliográficas