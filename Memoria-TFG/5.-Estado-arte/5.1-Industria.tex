La industria de la web3.0 se compone de una variedad de proyectos y empresas
que están trabajando en diferentes aspectos de esta tecnología. Algunos de
estos proyectos incluyen:

\subsection*{Protocolos Blockchain}
Son sistemas que permiten la creación y operación de redes descentralizadas y
seguras. Estos protocolos permiten la creación de aplicaciones
descentralizadas, contratos inteligentes y tokens criptográficos. Los
protocolos blockchain están en constante evolución y crecimiento, y están
impulsados por la comunidad y la colaboración. Algunos de los protocolos
blockchain más populares son Bitcoin, Ethereum, Binance Smart Chain, Polkadot y
Solana.

\subsection*{Criptomonedas}
Son monedas digitales que se basan en la tecnología blockchain y se utilizan
para realizar transacciones en línea de forma segura. Dentro de cada prtocolo
podemos encontrar distintas Criptomonedas, como pueden ser Bitcoin, Etheremun,
Dólares Theter, Solana, Shiba Inu, etc.

\subsection*{Aplicaciones descentralizadas (dApps)}
Este apartado se puede ver de forma más extendida dentro del subcapítulo
\ref{dApps}. \\ Son aplicaciones web que se ejecutan en la blockchain y
utilizan contratos inteligentes para permitir transacciones seguras y sin
intermediarios. Las dApps pueden ser utilizadas para una amplia variedad de
aplicaciones, desde finanzas descentralizadas (DeFi) hasta juegos y redes
sociales. Algunos ejemplos de dApps populares incluyen:
\begin{itemize}
    \item \textbf{Uniswap}: es una plataforma de intercambio descentralizada (DEX)
          que permite a los usuarios intercambiar criptomonedas sin la necesidad de
          intermediarios.
    \item \textbf{Decentraland}: es un mundo virtual descentralizado donde los
          usuarios pueden comprar, vender y construir propiedades virtuales utilizando
          criptomonedas.
    \item \textbf{Brave}: es un navegador web descentralizado que permite a los
          usuarios controlar su privacidad y monetizar su atención en línea.
    \item \textbf{OpenSea}: es un mercado de intercambio descentralizado para
          tokens criptográficos no fungibles (NFT) que permite a los usuarios comprar
          y vender obras de arte digitales, coleccionables y otros activos digitales
          únicos.

\end{itemize}

\subsection*{Conclusión}
La industria de la web3.0 está en constante evolución y crecimiento, y está
impulsada por la comunidad y la colaboración. Los proyectos y empresas en esta
industria están trabajando juntos para construir una internet más segura, justa
y descentralizada.