La razón principal que se encontró para la elaboración de este trabajo ha sido 
el auge de la Blockchain durante estos últimos años. "Blockchain es un libro 
mayor compartido e inalterable que facilita el proceso de registro de 
transacciones y de seguimiento de activos en una red de negocios. Un activo 
puede ser tangible o intangible. Prácticamente cualquier cosa de valor, puede 
rastrearse y comercializarse en una red de blockchain, reduciendo así el 
riesgo y los costes para todos los involucrados"\cite{ibmWeb3}. 

Esta tecnología ofrece la posibilidad de cosntruir aplicaciones web que garanticen
la privacidad y seguridad de los datos de un usuario. La Web 3 ofrece transparencia
y trazabilidad a la hora de la interacción entre usuarios. Además, permite la 
creación o conversión de negocios ya afianzados a este entorno.

Además, se afianza una necesidad que encontraba yo, personalmente, a la hora de
interactuar entre distintas carteras, que es la necesidad de tener todas en una 
misma ubicación.

Se debe constar que ya existen páginas afianzadas para el fin de mostrar y 
compra-venta de NFT's. Este trabajo busca el aprendizaje tanto del desarrollo 
de una aplicación web completa, como de aprender a realizar las llamadas y 
conexiones para recoger y mostrar datos de NFT's\cite{bbcNFTs} y 
criptocarteras\cite{moralisWallets}.