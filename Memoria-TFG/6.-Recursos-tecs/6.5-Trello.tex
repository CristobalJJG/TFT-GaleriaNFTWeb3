Trello es una herramienta de gestión de proyectos basada en la nube, que
permite a los usuarios organizar tareas y proyectos en tableros visuales y
colaborativos. Fue lanzado en 2011 y se ha vuelto muy popular entre equipos de
trabajo de diversas industrias y tamaños.
\begin{figure}[htb!]
    \centering
    \caption{Logo de Trello}
    \label{fig:trello-logo}
    \centering
    \includegraphics[scale=0.05]{./Ilustraciones/logos/trello.1024x1024.png}\\
    \textbf{Fuente:} Iconduck [\url{https://iconduck.com/icons/95002/trello}]
\end{figure}
\hfill \break
Trello utiliza un sistema de tableros que representan los diferentes proyectos
o áreas de trabajo, y dentro de cada tablero, los usuarios pueden crear listas
de tareas y tarjetas que representan cada tarea o actividad. Estas tarjetas
pueden contener información como descripciones, listas de verificación,
etiquetas, fechas de vencimiento, comentarios y archivos adjuntos.

Además, Trello permite la colaboración entre equipos de trabajo, ya que los
miembros pueden comentar en las tarjetas, asignar tareas a otros miembros,
establecer fechas de vencimiento y recibir notificaciones de actualizaciones.
Trello también se integra con otras herramientas populares de productividad,
como Slack, Google Drive y Jira, lo que lo hace aún más útil para equipos de
trabajo que utilizan diferentes herramientas.