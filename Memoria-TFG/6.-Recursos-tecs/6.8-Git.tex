\section{Git}
La información de este apartado se ha obtenido desde la página de Atlassian
    [\url{https://www.atlassian.com/git}] \\

Git es un sistema de control de versiones gratuito y de código abierto, creado
originalmente por Linus Torvalds en 2005. Con Git, cada desarrollador tiene el
historial completo de su repositorio de código localmente. Esto hace que la
clonación inicial del repositorio sea más lenta, pero las operaciones
posteriores, como confirmar, diferenciar, fusionar y registrar, son mucho más
rápidas.
\begin{figure}[htb!]
    \centering
    \caption{Logo de Figma}\label{fig:git-logo}
    \centering
    \includegraphics[scale=0.25]{./Ilustraciones/logos/git-logo.png}\\
    \textbf{Fuente:} Iconduck [\url{https://iconduck.com/icons/27401/git}]
\end{figure}

Git también tiene un excelente soporte para bifurcar, fusionar y reescribir el
historial del repositorio, lo que ha dado lugar a muchos flujos de trabajo y
herramientas innovadoras y potentes. Las solicitudes de incorporación de
cambios son una de esas herramientas populares que permiten a los equipos
colaborar en ramas de Git y revisar de manera eficiente el código de los demás.
Git es el sistema de control de versiones más utilizado en el mundo actual y se
considera el estándar moderno para el desarrollo de software.

Las órdenes básicas de Git son:
\begin{itemize}
    \item git init: crea un nuevo repositorio de Git vacío en el directorio actual.
    \item git clone: clona un repositorio existente en un nuevo directorio.
    \item git add: agrega cambios al área de preparación (staging area) para que estén
          listos para ser confirmados.
    \item git commit: crea un nuevo commit (instantánea) con los cambios agregados al
          área de preparación y agrega un mensaje de confirmación que describe los
          cambios.
    \item git status: muestra el estado actual del repositorio, incluyendo los cambios
          sin confirmar y los archivos sin seguimiento.
    \item git log: muestra una lista de todos los commits en orden cronológico inverso.
    \item git pull: actualiza el repositorio local con los cambios más recientes del
          repositorio remoto.
    \item git push: envía los cambios locales al repositorio remoto.
    \item git branch: muestra una lista de todas las ramas en el repositorio.
    \item git checkout: cambia a otra rama o commit.
\end{itemize}

\section{GitKraken}\label{git-kraken}
GitKraken es una herramienta de gestión de versiones de código que ofrece una
interfaz visual y fácil de usar para trabajar con Git. Es una aplicación de
escritorio que permite a los desarrolladores y equipos de desarrollo colaborar
en proyectos de software de manera eficiente.

Con GitKraken, se pueden realizar operaciones comunes de Git, como hacer
commits, fusionar ramas, crear y clonar repositorios, y gestionar conflictos de
fusión.

Además, cuenta con características adicionales, como la posibilidad de
integrarse con plataformas de gestión de proyectos como Trello, la capacidad de
ver el historial de cambios en el código, y la opción de visualizar y comparar
ramas de Git.

Grandes empresas hacen uso de esta herramienta, empesas tales como Netflix,
Philips, Amazon, Unity y Disney entre otras.

\begin{figure}[htb!]
    \centering
    \caption{Logo de Figma}\label{fig:gitkraken-logo}
    \centering
    \includegraphics[scale=0.15]{./Ilustraciones/logos/gitkraken-logo.png}\\
    \textbf{Fuente:} Iconduck [\url{https://iconduck.com/icons/27407/gitkraken}]
\end{figure}