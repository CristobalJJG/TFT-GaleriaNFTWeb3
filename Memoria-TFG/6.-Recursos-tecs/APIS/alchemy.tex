Alchemy es una plataforma de desarrollo con soporte multicadena y con alcance
global, pensada en facilitar el desarrollo de aplicaciones descentralizadas
(DApps). Su principal objetivo es ofrecer todo lo que los desarrolladores
necesitan para construir y hacer realidad la Web3.[\cite{alchemy}]
\begin{figure}[htb!]
    \centering
    \caption{Logo de Alchemy}
    \label{fig:alchemy-logo}
    \centering
    \includegraphics[scale=0.5]{./Ilustraciones/logos/alchemy-logo.png}\\
    \textbf{Fuente:} Página oficial de Alchemy [\url{https://www.alchemy.com}]
\end{figure}
\hfill \break

Su relevancia en el sector le ha válido ser reconocida como el “AWS de la
Web3”, manejando más de 10 millones de usuarios, movilizando más de 100 mil
millones de dólares en activos digitales y con un concurrencia de más de 100
mil millones de requests, lo que le ha llevado a tener una valoración de
mercado de más 10 mil millones de dólares.\\

La intención de la plataforma es permitir que las aplicaciones puedan
evolucionar rápidamente para dar respuesta a las necesidades de los usuarios,
sin que esto implique la puesta en marcha de tales mejoras. Así, los
desarrolladores se pueden enfocar en lo realmente importante: diseñar y
codificar estas nuevas soluciones, confiando en que la plataforma tendrá la
flexibilidad necesaria para respaldar estos nuevos diseños y permitir que los
usuarios puedan explorarlos.