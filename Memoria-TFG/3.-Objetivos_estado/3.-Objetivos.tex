El principal objetivo de este Trabajo de Fin de Grado se centra en el desarrollo
de un sitio que que hace uso del nuevo concepto de descentralización dentro del
marco de la web 3.0 (Ver la evolución de la web en la sección \ref{txt:web3}).

El resultado ofrece como funcionalidad principal la búsqueda de NFT's (Ver la 
descripción en la sección \ref{txt:nft}) y el 
almacenamiento de distintas carteras de criptomonedas (Ver la 
descripción en la sección \ref{txt:wallet}).

Para ello se plantea un conjunto de objetivos generales:
\begin{enumerate}
    \item Estudiar las características y evolución de la Web 3.0 y su relación con la Web3.
    \begin{itemize}
        \item Como se puede ver en el Capítulo 4, se han estudiado tanto las características, como la 
        evolución de las Web3, sus diversos usos en la actualidad.
    \end{itemize}
    \item Realizar un estudio de las tecnologías y herramientas a emplear.
    \begin{itemize}
        \item Este apartado se comprobará mejor en el Capítulo 
        \item Firebase es una de las principales herramientas de Google para el manejo de usuarios
        \item Hay una gran cantidad de API's que nos ayudan a acceder a la Web3, entre ellas:
        \begin{itemize}
            \item Coinbase: \url{https://docs.cloud.coinbase.com}
            \item ThirdWeb: \url{https://thirdweb.com}
            \item Moralis: \url{https://moralis.io}
            \item Alchemy: \url{https://www.alchemy.com}
        \end{itemize} 
        \item Para el backend se revisó el uso de Django y de Express.js:
        \begin{itemize}
            \item Usé Django en las Prácticas Externas y pude ver que es de fácil 
            uso y comprensión debido al uso de Python, pero eran necesarios demasiados
            pasos para el despliegue de la aplicación
            \item Con Express.js fue simple, ya que ejecutar un servidor que escuche un 
            puerto en concreto, es la base del funcionamiento de este Framework
        \end{itemize} 
        \item Para desplegar la página se ha revisado el funcionamiento de Amazon Web 
        Service, GitHub Pages y Vercel:
        \begin{itemize}
            \item Amazon Web Services es multiusos, lo vi como "matar una mosca 
            a cañozazos", algo demasiado grande para lo que iba a ser el proyecto
            \item A la hora de desplegar un proyecto angular en GitHub Pages me 
            resultó complicado, ya que se tenía que crear la build del 
            proyecto, y luego asignar el fichero que se mostraba. 
            Si en algún momento me olvidaba de hacer una build, no se actualizaba 
            aquí, por lo que fue descartada
            \item Vercel aportaba algo simple, ya que si era un proyecto Angular 
            lo desplegaba directamente, y a la hora de desplegar el backend, hecho en Express.js 
            resultó ser sencillo, con un fichero de configuración escrito gracias a la 
            documentación de Vercel 
        \end{itemize} 
    \end{itemize}
    \item Establecer criterios para su selección (tanto las tecnologías como de las herramientas)
    \begin{itemize}
        \item Debido al "poco tiempo de desarollo" decidí escoger las herramientas según 
        la complejidad de las mismas:
        \begin{itemize}
            \item Para tener la información de los usuarios, las distintas 
            colecciones a mostrar, etc. Se usó Firebase por ser de Google y 
            de las más completas
            \item Para la obtención de NFT's se hizo uso de la API Alchemy, 
            que con la documentación que se encontraba en la web se hizo fácil de usar
            \item Para la obtención del balance de las carteras a partir de su 
            dirección pública se hizo uso de la API de Moralis, ya que pude encontrar 
            diversas formas de hacerlo, pero esta fue la más efectiva
            \item Para esta última parte se hizo un backend con Express.js, que nos 
            permite hacer uso de la funcionalidad de JavaScript para hacerlo
            \item Para el despliegue de la aplicación se hizo uso de Vercel, por su 
            simplicidad a la hora de desplegar los proyectos y documentación 
        \end{itemize} 
    \end{itemize}
    \item Seguir una metodología de desarrollo, las pruebas y documentación del proyecto.
\end{enumerate}